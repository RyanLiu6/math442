\documentclass[12pt,a4paper]{article}
\usepackage{amsmath,amssymb,amsthm,epsf, graphicx, rotating}
\usepackage{fancyhdr}
\usepackage{subfig}
\usepackage{float}

\pagestyle{empty}
\setlength{\parindent}{0pt}
\setlength{\textwidth}{6.5in}
\setlength{\oddsidemargin}{0in}
\addtolength{\textheight}{1in}

\renewcommand\theenumi{\alph{enumi}}
\renewcommand\labelenumi{(\theenumi)}

\newcommand{\Z}{\mathbb{Z}}
\newcommand{\F}{\mathbb{F}}
\newcommand{\R}{\mathbb{R}}
\newcommand{\C}{\mathbb{C}}
\newcommand{\N}{\mathbb{N}}
\renewcommand\vec{\mathbf}

\pagestyle{fancy}
\fancyhf{}
\fancyhead[LE, RO]{Ryan Liu}

\theoremstyle{definition}
\newtheorem{problem}{}

\author{Ryan Liu}
\title{MATH 442 Homework 8}
\begin{document}

\begin{center}
{\huge MATH 442 \par}
{\Large Homework  8  \par}
{\normalsize Name: Ryan Zhuo Lun Liu \par}
{\normalsize Student Number: 30328141 \par}
{\normalsize Collaborator: Robert Benjamin Lang \par}
\end{center}

\begin{problem} \underline{Answer:}
\begin{proof} 
The coefficient of $k^n$ will be proved directly.\\

Consider the vertices of $G$, and whether they are connected or not, will contribute to $P_G(k)$. At most, they will contribute $k$ possibilities, and at worst, will contribute $k - n + 1$ possibilities. In either case, $P_G(k)$, the product of the possibilities for all vertices would produce a monic polynomial - the polynomial where the leading coefficient is 1. \\

The coefficient of $k^{n - 1}$ will be proved by strong induction on the number of edges, $e$. \\

\underline{Base Case:} $e = 0$ and $P_G(k) = k^n$, which satisfies the condition. \\
\underline{Inductive Step:} Suppose the statement is true for all graphs with $\leq e$ edges. Let $G = (V, E)$ be a graph with $e + 1$ edges. \\

Let $x \in E$ and consider the graphs $G - x$ and $G/x$. Note that $G - x$ has $e$ edges and $G/x$ has $n - 1$ vertices and at most $e$ edges. Applying the inductive hypothesis gives us: \\

$P_{G - x}(k) = k^n - (e)k^{n - 1} + ...$ \\
$P_{G/x}(k) = k^{n - 1} + + ...$ \\

By the Deletion-Contraction Theorem, we know that $P_G(k) = P_{G - x}(k) - P_{G/x}(k)$: \\

$P_G(k) = k^n - (e + 1)k^{n - 1} + ...$ \\

Therefore, by strong induction, the result holds. 
\end{proof}
\end{problem}

\begin{problem} \underline{Answer:}
\begin{proof}
Consider a simple graph $G$ that is disconnected and that $G$ has $N$ distinct connected components, and let $H$ represent the set of connected components. We see that for any $h \in H$, we can colour $h$ regardless of the colouring of the other connected components. \\

Thus, $P_G(k) = \prod_{i = 1}^{N} P_{h_i}(k)$, or the product of the chromatic polynomial of $G$'s connected components.
\end{proof}
\end{problem}

\begin{problem} \underline{Answer:} 
\begin{proof} The statement, $\chi(G) = n$ iff $G = K_n$ will be proved directly both ways. \\

$\rightarrow$ Suppose that $\chi(G) = n$, and consider a graph $G$ with $n$ vertices. Since the chromatic number of $G$ is $n$, then at minimum, it takes $n$ colours to colour $G$. This would only occur if every vertex is adjacent to every other vertex, forcing us to use $n$ colours. This is also followed directly by Brook's Theorem. \\

$\leftarrow$ Suppose $G = (V, E) = K_n$ and consider a vertex $v \in V$. Since it is the complete graph, $v$ must be adjacent to every other vertex $u \in V - v$, meaning $v$ has a degree of $n - 1$. Now, suppose $v$ is coloured $c_1$. In order for $G$ to have a proper colouring, every vertex adjacent to $v$ must be coloured $c_1$. But as $G = K_n$, the above scenario extends to all other vertices and we would need $n$ colours to properly colour $G$.
\end{proof}
\end{problem}

\begin{problem} \underline{Answer:} 
\begin{proof} This will be proved directly using the Binomial Expansion. \\

The chromatic polynomial, $P_G(k)$, is the product of the possible ways to colour a graph. One way to represent this is $(k - 0)(k - 1)(k - 2) ... (k - n + 1)$. This can also be presented by $\prod_{i = 0}^{n + 1} k - i$. 
As we are only concerned with the signs of the chromatic polynomial, we can disregard $i$ and simplify the polynomial to $(k - \beta)^n$. Let's call this $P_{G_s}(k)$, the sign chromatic polynomial. \\

We can expand $(k - \beta)^n$ using the binomial expansion to give us $\sum_{i = 0}^{n} {{n \choose i} (k^{n - i})(-\beta)^i} = \sum_{i = 0}^{n} {{n \choose i} (k^{n - i})(\beta)^i(-1)^i}$. \\

We see that each term in the expansion has $(-1)^i$, which alternates in sign. Therefore, each term in $P_G(s)$ also alternates in sign.
\end{proof}
\end{problem}

\begin{problem} \underline{Answer:} 
\begin{proof} Following the proof of the five colour theorem from class, the proof for four colour theorem breaks at one of the first assumption. \\

In the proof of the five colour theorem, we assumed that every simple planar graph has at least one vertex of at most degree $5$. However, we cannot directly assume this for the four colour theorem and state that every simple planar graph has at least one vertex of at most degree $4$.
One contradiction of the assumption is the icosahedral graph, which is a simple planar graph that does not have a vertex of at most degree $4$.
\end{proof}
\end{problem}

\begin{problem} \underline{Answer:}
\begin{proof} The statement, if $G$ is $k$-critical then every vertex has degree at least $k - 1$ will be proved by contradiction. \\

Suppose that $G$ is $k$-critical and that it has a vertex $v$ whose degree is $< k - 1$. If we remove $v$, the graph $G - v$ is $k - 1$ colourable as each vertex has degree at least $k - 1$. \\

Next, we want to add $v$ back into the graph, and since $v$ has degree $< k - 1$, there is at least one free colour we can assign it. Thus, the graph $G$ is $k - 1$ colourable. Hence, $G$ is not $k$-critical and we have arrived at a contradiction.
\end{proof}
\end{problem}
\end{document}
