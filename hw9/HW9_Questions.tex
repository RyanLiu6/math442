\documentclass[12pt]{article}
\usepackage{amsmath,amssymb,amsthm,epsf, graphics, rotating}
\usepackage{tikz}

\pagestyle{empty}
\setlength{\parindent}{0pt}
\setlength{\textwidth}{6.5in}
\setlength{\oddsidemargin}{0in}
\addtolength{\textheight}{1in}

\renewcommand\theenumi{\alph{enumi}}
\renewcommand\labelenumi{(\theenumi)}


\newcommand{\Z}{\mathbb{Z}}
\newcommand{\F}{\mathbb{F}}
\newcommand{\R}{\mathbb{R}}
\newcommand{\C}{\mathbb{C}}
\renewcommand\vec{\mathbf}

\theoremstyle{definition}
\newtheorem{problem}{}
\newtheorem{solution}{}
\title{\vspace{-2.0cm}Math 442 Homework 9}
\date{}

%\renewcommand{\bigskip}{\vfill}

\begin{document}
%\vspace*{-1.2in}
%
%
\maketitle
%
\vspace{-16 mm}
\begin{itemize}
\item Due Thursday March 21 at start of class.
%
\item If your homework is longer than one page, {\bf staple} the pages together, and put your name on each sheet of paper.
%
\item {\bf Collaboration Policy}: You are welcome (and encouraged) to work on the homework in groups. However, each student must write up the homework on their own, and must use their own wording (i.e.~don't jusy copy the solutions from your friend). If you do collaborate with others, please list the name of your collaborators at the top of the homework.

\item You are encouraged (though not required) to type up your solutions. If you choose to do this, I strongly recommend that you use the typesetting software LaTeX. LaTeX is used by the entire mathematics community, and if you intend to go into math, you’ll need to learn it sooner or later. ``The Not So Short Introduction to LaTeX'' is a good place to start. This guide can be found at http://tug.ctan.org/info/lshort/english/lshort.pdf . You can also download the .tex source file for this homework and take a look at that.

\item Each homework problem should be correct as stated. Occasionally, however, I might screw something up and give you an impossible homework problem. If you believe a problem is incorrect, please email me. If you are right, the first person to point out an error will get +1 on that homework, and I will post an updated version. 
\end{itemize}

\begin{problem}
Give an example of a graph that is all three of 2-chromatic, 2-chromatic(e) and 2-chromatic(f). Show briefly why.
\end{problem}

\begin{problem}
Give and prove an explicit edge colouring of $Q_k$ with $k$ colours, and hence prove that $\chi '(Q_k) = k$.
\end{problem}

\begin{problem}
Let $G$ be a simple graph with an odd number of vertices. Prove that if $G$ is regular of degree $d\geq 2$ then $\chi ' (G) = d+1$.
\end{problem}

\begin{problem} 
A lecture timetable is to be drawn up. Certain lectures must not coincide. The $\ast$s in the following table show which lectures must not coincide. How many lecture periods are needed to timetable all 7 lectures?

$$\begin{array}{r|ccccccc}
&a&b&c&d&e&f&g\\\hline
a&&\ast&\ast&\ast&&&\ast\\
b&\ast&&\ast&\ast&\ast&&\ast\\
c&\ast&\ast&&\ast&&\ast&\\
d&\ast&\ast&\ast&&&\ast&\\
e&&\ast&&&&&\\
f&&&\ast&\ast&&&\ast\\
g&\ast&\ast&&&&\ast&\end{array}
$$
\end{problem}

\begin{problem}
Prove a simple connected graph $T$ is a tree if and only if adding an edge between two existing vertices of $T$ creates exactly one cycle.
\end{problem}


\begin{problem}
Let $T$ be a tree with average degree $a$. Find and prove a formula for the number of vertices in $T$ in terms of $a$.
\end{problem}

\begin{problem}
Let $T$ be a tree  with at least two vertices and with no vertices of degree 2. Prove that $T$ has more leaves than non-leaf vertices.
\end{problem}
\end{document}