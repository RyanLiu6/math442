\documentclass[12pt]{article}
\usepackage{amsmath,amssymb,amsthm,epsf, graphics, rotating}
\usepackage{tikz}

\pagestyle{empty}
\setlength{\parindent}{0pt}
\setlength{\textwidth}{6.5in}
\setlength{\oddsidemargin}{0in}
\addtolength{\textheight}{1in}

\renewcommand\theenumi{\alph{enumi}}
\renewcommand\labelenumi{(\theenumi)}


\newcommand{\Z}{\mathbb{Z}}
\newcommand{\F}{\mathbb{F}}
\newcommand{\R}{\mathbb{R}}
\newcommand{\C}{\mathbb{C}}
\renewcommand\vec{\mathbf}

\theoremstyle{definition}
\newtheorem{problem}{}
\newtheorem{solution}{}
\title{\vspace{-2.0cm}Math 442 Homework 4}
\date{}

%\renewcommand{\bigskip}{\vfill}

\begin{document}
%\vspace*{-1.2in}
%
%
\maketitle
%
\vspace{-16 mm}
\begin{itemize}
\item Due Thursday January 31 at start of class.
%
\item If your homework is longer than one page, {\bf staple} the pages together, and put your name on each sheet of paper.
%
\item {\bf Collaboration Policy}: You are welcome (and encouraged) to work on the homework in groups. However, each student must write up the homework on their own, and must use their own wording (i.e.~don't jusy copy the solutions from your friend). If you do collaborate with others, please list the name of your collaborators at the top of the homework.

\item You are encouraged (though not required) to type up your solutions. If you choose to do this, I strongly recommend that you use the typesetting software LaTeX. LaTeX is used by the entire mathematics community, and if you intend to go into math, you’ll need to learn it sooner or later. ``The Not So Short Introduction to LaTeX'' is a good place to start. This guide can be found at http://tug.ctan.org/info/lshort/english/lshort.pdf . You can also download the .tex source file for this homework and take a look at that.

\item Each homework problem should be correct as stated. Occasionally, however, I might screw something up and give you an impossible homework problem. If you believe a problem is incorrect, please email me. If you are right, the first person to point out an error will get +1 on that homework, and I will post an updated version. 
\end{itemize}
\begin{problem}
 Show the Gr\"otzsch graph below is Hamiltonian.

\begin{figure}[h]\centerline{\includegraphics[scale=1]{2019figs/grotzsch.pdf}}
\end{figure}
\end{problem}

\begin{problem}
\begin{enumerate}
\item For which values of $n$ is the complete graph $K_n$ Eulerian? Hamiltonian?\item For which values of $m, n$ is the complete bipartite graph $K_{m,n}$ Eulerian? Hamiltonian?\end{enumerate}

Prove each of your answers.
\end{problem}

\begin{problem} 
\begin{enumerate}
\item Does every Eulerian bipartite graph have an even number of edges?
\item Does every Eulerian simple graph with an even number of vertices have an even number of edges?
\end{enumerate}


If yes then give a proof, and if no then give a counterexample.
\end{problem}



\begin{problem}
 Let $G$ be a Hamiltonian graph and let $S$ be any set of $k$ vertices in $G$. Prove that the graph $G-S$ has at most $k$ components.
\end{problem}





\begin{problem}
Prove that $Q_k$ for all $k\geq 2$ is Hamiltonian.
\end{problem}

Find and prove a formula for the number of vertices of $Q_k$. Find and prove a formula for the number of edges of $Q_k$. 


\end{document}